%!TEX root = pset1.tex

\section{Sparsity and LASSO}\label{sec:gen}
An alternative approach is to use the $L_1$ norm regularizer in the objective for regularized least squares.  The error function that we aim to minimize over is given by:
\begin{equation} \label{eq:ridge_error_fn}
\frac{1}{N} \sum_{n=1}^{N} (t_n - \M{w}^T \phi (\M{x}_n) )^2 + \frac{\lambda}{2} \sum_{j=1}^{M} |w_j|
\end{equation}

The LASSO is widely used in practice because it leads to solutions which are relatively sparse, i.e. the resulting prediction vector $\M{w}$ tends to have few non-zero components.  

To test this method and its sparsity properties, we consider a regression problem with 5 training points and 500 testing points in $\mathbb{R}^{12}$.  The data is generated according to  $f(x) = \M{w}^T \phi (\M{x}) + \epsilon$, where $\epsilon \sim  N(0,\sigma^2)$ and $\M{w} \in \mathbb{R}^{12}$ is a sparse vector with only two non-zero components.  For this problem, we assume that $\M{w}$ is unknown, and we estimate this vector with ridge regression in Section~\ref{sec:sparsity_ridge_reg} and with LASSO in Section~\ref{sec:sparsity_grad_desc}.  Let $\lambda_2$ denote the regularizer term for the $L_2$ norm and let $\lambda_1$ denote the regularizer term for the $L_1$ norm.  Figure~\ref{fig:highdim} shows plots of the estimated function with different regularizers.  

\subsection{Ridge Regression Gradient Descent} \label{sec:sparsity_ridge_reg}
Applying our method for gradient descent to the ridge regression problem on the training dataset of 12 points, we compute the vector of weights $\M{w}$ and MSE on the testing dataset for $\lambda_2 = 0.1$.  We obtain: 
%
\begin{flalign*}
\M{w} &= (w_1, w_2, w_3, w_4, w_5, w_6, w_7, w_8, w_9, w_{10}, w_{11}, w_{12}),\\
\text{MSE} &= TODO.
\end{flalign*}
%
If we disable the $L_2$ regularizer (set $\lambda_2 = 0$), we obtain:
%
\begin{flalign*}
\M{w} &= (w_1, w_2, w_3, w_4, w_5, w_6, w_7, w_8, w_9, w_{10}, w_{11}, w_{12}),\\
\text{MSE} &= TODO.
\end{flalign*}
%
\subsection{LASSO Gradient Descent} \label{sec:sparsity_grad_desc}
Similarly, we apply our method for gradient descent to the LASSO problem on the on the training dataset of 12 points.  For $\lambda_1 = 0.1$, we find: 
%
\begin{flalign*}
\M{w} &= (w_1, w_2, w_3, w_4, w_5, w_6, w_7, w_8, w_9, w_{10}, w_{11}, w_{12}),\\
\text{MSE} &= TODO.
\end{flalign*}
%
If we disable the $L_1$ regularizer (set $\lambda_1 = 0$), we find:
%
\begin{flalign*}
\M{w} &= (w_1, w_2, w_3, w_4, w_5, w_6, w_7, w_8, w_9, w_{10}, w_{11}, w_{12}),\\
\text{MSE} &= TODO.
\end{flalign*}
%
\subsection{Sparsity Properties of LASSO}